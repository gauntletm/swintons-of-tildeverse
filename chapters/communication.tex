Tilde servers are communities, and communities only grow and stay alive by means of communication and information interchange. Naturally, methods were established on all the servers to allow users to chit-chat, seek help and create new stuff.

\section{Mail}
With either pine or mutt installed, swintons are able to communicate with one another on their local machine. Hardly any server allows to send mail outwards or receive any from the internet.\\
In case a mail is received, the shell will inform the user after login.

\section{IRC}
The ancestor of today's instant messaging services and somehow still alive and well. Programs like scrollz or irssi allow to join the conversation right from the command line. Some classic shell providers even require their users to parttake in IRC as to keep the community connected.

\section{wall}
wall is kind of a large scale baseball bat that will indifferently strike anyone logged in on the server right in the face. Whatever the sender writes, the other users will see appear in their STDIN. As this is likely to mess up carefully edited scripts or ASCII art, wall is hardly used.

\section{Forums}
In February 2018 calamitous, admin of ctrl-c.club, introduced href{https://github.com/Calamitous/iris}{iris}, a CLI forum built with ruby. As of February 2018, the forum allows for threaded communication with more than two dozen points on the todo list. Development is ongoing.
