\section{The tilde sign}
It is the users that make a tilde server what it is.

On UNIX and derived systems like GNU/Linux, the $\sim$ sign always points to a users own home directory, which is the place on the machine where their personal files and configuration are stored. On web servers, the tilde may be used to indicate that the content being displayed belongs to a user.

Usually, all files stored in /home/username/public\_html/ are served to a web browser when it accesses www.example.com/$\sim$username. However, this is a matter of configuration and is not necessarily the case.

remotes.club, for example, made user pages available at username.remotes.club, entirely setting the tilde aside.
\\

Since computer people are somewhat lazy, $\sim$gschueler wrote some JavaScript to \href{http://tilde.club/~gschueler/tildelink.html}{autolink usernames}.

\section{Swintons}
In search of a succinct name for the users of tilde servers, \href{http://tilde.club/~cortex/}{$\sim$cortex} came up with the term Swinton (because tilde and \href{https://en.wikipedia.org/wiki/Tilda_Swinton}{Tilda} do sound somewhat similar).

The term 'tildenizens', as proposed \href{http://tilde.club/~joeld/tildelore.html}{here}, did apparently not meet approval.\\

Since tilde users usually are quite knowledgeable about the internet and the web, terms like \href{https://en.wikipedia.org/wiki/Netizen}{netizen} or \href{https://en.wikipedia.org/wiki/Internaut}{intenaut} might also apply, with swintons as a subcategory.

\section{Meetups}
I know of two cases in which swintons met up, in both cases in October 2014 and related to tilde.club, while meetups partaining to other servers seemingly did not happen yet.\\

The \href{http://tilde.club/~joeld/minnesota.html}{Minnesotans} $\sim$ met on \href{http://tilde.club/~joeld/#523}{October 15} at Brit’s Pub and even took a photo.\\

A photo was \href{http://tilde.club/~cortex/#0019}{taken} by $\sim$agray and $\sim$cortex, too, when they met at 33 Acres in Vancouver on October 20.
\section{.plan}
Each user may store a file called .plan in their home directory. This \href{https://en.wikipedia.org/wiki/Hidden_file_and_hidden_directory#Unix_and_Unix-like_environments}{dotfile} may contain arbitrary information about the user, like an external email address, a twitter handel, URLs for other web presences et cetera.\\

The program \href{https://en.wikipedia.org/wiki/Finger_protocol}{finger} may then be used to gather information about any user on the local machine and read their plan file.\\

In theory, finger may be used via TCP/IP to request plan files and information on users on remote machines, but the required service does not seem to be activated on any tilde server.

\section{Maps}
Fellow swinton $\sim$bear was busy creating maps of user locations on different servers, e.g. on \href{}{}, \href{https://tilde.red/~bear/where.html}{tilde.red} and \href{http://tilde.town/~bear/where.html}{tilde.town}.

He released the code used to collect the data, store it in .json and make up the actual maps in a \href{https://github.com/thebaer/tildes/blob/master/where/where.go}{GitHub repository}.
\\

A similar thing was done by $\sim$eric on \href{http://tilde.club/~eric/locations.html}{tilde.club}.
