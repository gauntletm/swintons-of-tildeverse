There was a time before big players like Facebook, Google or Twitter hoarded users on their servers. A time before those virtual mega cities existed.

A time in which small-ish UNIX servers made up the web and users logged in on their servers on a command line interface to write and host their personal websites there. But with the rise of blogging sites and social networks, most people chose the easy way.

Many didn't even know of these \href{http://www.loomcom.com/blog/2014/10/13/the-1993-social-network/}{antique social networks} of the eighties and nineties, much less of the existence of such a thing as the command line.

The world wide web had become an easy to use, colorful, graphical adventure and left its text adventure times behind, with only enthusiasts (and, of course, admins) knowing the ways of the past.
\\
\\
When Paul Ford read \href{http://rockalittle.com/thanksgiving2004.htm}{this article} in september 2014, it was the $\sim$ sign that caught his attention. Following a short mention on his twitter account he got a little insight on what he had stumbled upon.

\begin{displayquote}
tildes are only ever properly used in front of usernames on shared hosting.\\
-- \href{https://twitter.com/mathowie/status/516776784868548608}{@mathowie}, 2014-09-29
\end{displayquote}

Apparently, it took Paul only a few moments to do a bit more research on this matter and register a domain name for his new project based on this finding, because seven minutes later he announced the creation of tilde.club and invited users to join him on the server.

\begin{displayquote}
\href{https://twitter.com/mathowie/}{@mathowie} YES also i just registered \href{http://tilde.club}{http://tilde.club}  and will given anyone a shell account who wants one\\
-- \href{https://twitter.com/ftrain/status/516778575828381696}{@ftrain}, 2014-09-29
\end{displayquote}

Short time after, other twitter users began to ask for accounts. As Paul himself said, he had 100 requests when he woke up the next day. He set up the accounts and watched his server grow until it reached its capacities and he had to queue new requests on a wait list.

The outcome was \href{http://www.tilde.club/}{tilde.club}, a remote server people could log in on via SSH.\\
The possibilities were numerous. Most users used their $\sim$\/public\_html  directory to create web pages. Others used the tools on board to create games, write fiction et cetera. More than just a few started blogging.

\begin{displayquote}
It was humans interacting with humans. It was people playing around, experimenting.\\
-- \href{http://www.thedailybeast.com/articles/2015/12/02/eating-bread-and-crying-on-the-floor-lonely-guy-finds-34k-new-year-s-dates.html}{Ben Collins}, 2015-02-12
\end{displayquote}

Servers of this kind existed long ago and have been available to this day, like the time-honored \href{http://sdf.org/}{SDF Public Access UNIX System}, \href{https://st0rage.org/}{st0rage.org} and \href{http://shells.red-pill.eu/}{dozens more}. They worked the same way, they offered the same functions and they, too, did not (necessarily) cost the users any money. So what made tilde.club so special?\\

Perhaps because was the farily strong difference to the web that exists today. Tilde servers are by far not as polished and easy to use as modern social networks while also being more playful and exciting than the established shell providers. Well-tried Linux users and rookies alike could explore the remote machine with eyes shining with joy and build things without pressure or competion.

In the days and weeks after the creation of tilde.club, other people followed the idea and started their own servers. These are, according to \href{https://tilde.town/~nossidge/tildeverse/boxes.html}{$\sim$nossidge} and \href{http://tilde.club/~pfhawkins/othertildes.html}{$\sim$pfhawkins}, the 38 tilde servers that appeared until now in alphabetic order:
\begin{multicols}{2}
	\begin{itemize}
		\item \href{https://bleepbloop.club/}{bleepbloop.club}
		\item \href{https://botb.club/}{botb.club}
		\item \href{http://catbeard.city/}{catbeard.city}
		\item \href{https://club6.nl}{club6.nl}
		\item \href{http://ctrl-c.club/}{Ctrl-C Club}
		\item \href{http://cybyte.club/}{cybyte club}
		\item \href{http://drawbridge.club/}{drawbridge.club}
		\item \href{http://germantil.de/}{german tilde}
		\item \href{http://hackers.cool/}{hackers.cool}
		\item \href{http://hypertext.website/}{hypertext.website}
		\item \href{http://losangeles.pablo.xyz/}{losangeles.pablo.xyz}
		\item \href{http://matilde.club/}{matilde.club}
		\item \href{http://noiseandsignal.com/}{noiseandsignal.com}
		\item \href{http://oldbsd.club/}{oldbsd.club}
		\item \href{http://palvelin.club/}{palvelin club}
		\item \href{http://pebble.ink/}{pebble.ink}
		\item \href{http://perispomeni.club/}{perispomeni.club}
		\item \href{http://protocol.club/}{protocol club}
		\item \href{https://remotes.club/}{remotes.club}
		\item \href{http://retronet.net/}{retronet}
		\item \href{http://riotgirl.club/}{RIOTGIRL.CLUB}
		\item \href{http://rudimentarylathe.org/}{rudimentarylathe.org}
		\item \href{https://spookyscary.science/}{Scary Spooky Scary}
		\item \href{http://skylab.org/}{skylab.org}
		\item \href{http://squiggle.city/}{squiggle city}
		\item \href{http://sunburnt.country/}{sunburnt.country}
		\item \href{http://tilde.camp/}{tilde.camp}
		\item \href{https://tilde.center/}{tilde.center}
		\item \href{http://tilde.city/}{tilde.city}
		\item \href{http://tilde.club/}{tilde.club}
		\item \href{http://tilde.farm/}{tilde.farm}
		\item \href{http://tilde.red/}{tilde.red}
		\item \href{http://tilde.town/}{tilde.town}
		\item \href{http://tilde.works/}{tilde.works}
		\item \href{http://tildesare.cool/}{tildesare.cool}
		\item \href{http://totallynuclear.club/}{totally nuclear club}
		\item \href{https://www.v3ctor.club/}{v3ctor.club}
		\item \href{http://yester.host/}{yester.host}
	\end{itemize}
\end{multicols}

By the way, club6.nl \href{http://tildesare.cool/~imt/2015/01/04/club6-nl-is-up.html}{claimed} to be the public access UNIX system accessible only via IPv6.