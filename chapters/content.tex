\section{Style}
The home page of tilde.club is all amber and black, a reminiscence of old times when computers were slow, loud and by far not as colorful. It is free of images and uses a simple, table style design.\\
Other tilde home pages like botb.club, ctrl-c.club or germantil.de borrowed design ideas from this page, sometimes altering the colors to green or purple.\\
Sites like tilde.red or riotgirl.club had their own apporaches, being either way more simple or way more colorful.\\
tilde.town was even mentioned on a web site about \href{http://brutalistwebsites.com/}{web brutalism}, representative for just about all tilde servers.
\\

Ugly and hard to use is a description that might apply to a few index pages and more than just a few user pages. Some swintons, however, dug deep into CSS and HTML and created beauty.\\
Others, not so much, and many pages are black on white with blue hyperlinks and more often than not in a monospace font.

\section{Collaboration}
The \href{https://github.com/tildetown/zine}{tilde.town zine} is a great example of the collaboration between tilde users. Even though activity is seemingly remaining static, the first issue shows the degree of creativity and and vigor in your average swinton.
\\

The report at hand will, as I hope, turn out as another example.
\\

\section{Art}
\begin{itemize}
	\item \href{http://tilde.town/~owenversteeg/}{ASCII shrimp}
	\item
\end{itemize}

\section{Dark Tilde}
$\sim$\_ \href{http://tilde.club/~_/index.html#october222014}{wondered} about stuff on tilde servers that is not linked to from the tilde server and in which degree this might qualify as a Dark Web equivalent. So, is there a Dark Tilde?\\
It usually does not last long until new projects are linked to, but sometimes they happen in the filesystem on the tilde somewhere off the areas a web server can access, so they may only be mentioned and will stay invisible for anyone without an account on the respective system.\\
Dark Tilde, in a way.
