The first users applied via twitter, responding directly to Paul Ford. Later, he set up a Google Form for people to use.\\
Other servers took up this method, some also allowed requests via email.

The Google Forms usually asked for the desired and an alternative username in case the first was already taken, required an email address and demanded approval of the the server rules.

While there is a full blown \href{http://www.tilde.town/~wiki/conduct.html}{code of conduct for tilde.town}, most servers went by the simple rules of being kind to one another and imperative of thinking about what the users did. Again inspired by Paul Ford, this rule was summed up in the words NO DRAMA, an explicit reference  to \href{https://en.wikipedia.org/wiki/No_More_Drama_(song)}{No More Drama} by Mary J. Blige.

Plus, sometimes, it was requested to not hack the gibson.

Some forms also ask for the SSH public key, creating a first obstacle for people not so fond of computers, but usually link to \href{https://github.com/tildeclub/tilde.club/blob/master/docs/ssh.md}{documentation} on this topic.

Additional fields, though not required, include but aren't limited to:
\begin{itemize}
	\item gender (riotgirl.club)
	\item twitter handle (dito)
	\item plans and interests (tilde.town)
\end{itemize}

Creation of the account includes the setup of the public\_html directory including the first index.html file, usually asking to be edited after login.

After the the account is created, which, depending on the server, might last only hours or never happen, the respective admin usually sends an email out to the new users, welcoming them on the server.

Either on that occasion or at login, by means of the \href{https://en.wikipedia.org/wiki/Motd_(Unix)}{motd}, users are pointed to the servers FAQ or a primer of some kind, like \href{http://tilde.club/~anthonydpaul/primer.html}{how to tilde} on tilde.club.


